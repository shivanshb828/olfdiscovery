\section{DISCUSSION}
If you are using \textit{Word}, use either the Microsoft Equation Editor or the \textit{MathType} add-on (\textit{http://www.mathtype.com}) for equations in your paper (Insert | Object | Create New | Microsoft Equation or MathType Equation). “Float over text” should \textit{not} be selected. 	

\subsection{Equations}
Number equations consecutively with equation numbers in parentheses flush with the right margin, as in (1). First use the equation editor to create the equation. Then select the “Equation” markup style. Press the tab key and write the equation number in parentheses. To make your equations more compact, you may use the solidus ( / ), the exp function, or appropriate exponents. Use parentheses to avoid ambiguities in denominators. Punctuate equations when they are part of a sentence, as in

\begin{equation}
\begin{multlined}
\int_{0}^{r_{2}} F(r,\varphi)dr d\varphi=[\sigma r_{2}/(2\mu_{0})]\\
\int_{0}^{+\infty}exp(-\lambda |z_{j}-z_{i} |)\lambda^{-1} J_{1}(\lambda r_{2}) J_{0}(\lambda r_{1})d\lambda 
\label{Eq:Example}
\end{multlined}
\end{equation}

Be sure that the symbols in your equation have been defined before the equation appears or immediately following. Italicize symbols (T might refer to temperature, but T is the unit tesla). Refer to “(1),” not “Eq. (1)” or “equation (1),” except at the beginning of a sentence: “Equation (1) is ... .”
A general IEEE styleguide is available at \textit{http://www.ieee.org/web/publications/authors/transjnl/ \\index.html}
